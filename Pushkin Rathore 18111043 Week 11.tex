% Options for packages loaded elsewhere
\PassOptionsToPackage{unicode}{hyperref}
\PassOptionsToPackage{hyphens}{url}
%
\documentclass[
]{article}
\usepackage{lmodern}
\usepackage{amssymb,amsmath}
\usepackage{ifxetex,ifluatex}
\ifnum 0\ifxetex 1\fi\ifluatex 1\fi=0 % if pdftex
  \usepackage[T1]{fontenc}
  \usepackage[utf8]{inputenc}
  \usepackage{textcomp} % provide euro and other symbols
\else % if luatex or xetex
  \usepackage{unicode-math}
  \defaultfontfeatures{Scale=MatchLowercase}
  \defaultfontfeatures[\rmfamily]{Ligatures=TeX,Scale=1}
\fi
% Use upquote if available, for straight quotes in verbatim environments
\IfFileExists{upquote.sty}{\usepackage{upquote}}{}
\IfFileExists{microtype.sty}{% use microtype if available
  \usepackage[]{microtype}
  \UseMicrotypeSet[protrusion]{basicmath} % disable protrusion for tt fonts
}{}
\makeatletter
\@ifundefined{KOMAClassName}{% if non-KOMA class
  \IfFileExists{parskip.sty}{%
    \usepackage{parskip}
  }{% else
    \setlength{\parindent}{0pt}
    \setlength{\parskip}{6pt plus 2pt minus 1pt}}
}{% if KOMA class
  \KOMAoptions{parskip=half}}
\makeatother
\usepackage{xcolor}
\IfFileExists{xurl.sty}{\usepackage{xurl}}{} % add URL line breaks if available
\IfFileExists{bookmark.sty}{\usepackage{bookmark}}{\usepackage{hyperref}}
\hypersetup{
  hidelinks,
  pdfcreator={LaTeX via pandoc}}
\urlstyle{same} % disable monospaced font for URLs
\setlength{\emergencystretch}{3em} % prevent overfull lines
\providecommand{\tightlist}{%
  \setlength{\itemsep}{0pt}\setlength{\parskip}{0pt}}
\setcounter{secnumdepth}{-\maxdimen} % remove section numbering

\author{}
\date{}

\begin{document}

\textbf{{Radiopharmaceuticals: Cancer Therapy}}

Pushkin Rathore 18111043

Week 11

April 1\textsuperscript{st}, 2022

\textbf{AIM}

The aim of this is paper is to make aware the reader about the
technologies that exists that make the treatment of certain diseases
such as cancer to be more efficient by the use of different approaches.
In this particular case, we'll look into the dealings in regards to the
radioactive substances and the science behind them that is being applied
to treat diseases such as cancer. Radiopharmaceuticals, or medicinal
radio compounds, are a group of pharmaceutical drugs containing
radioactive isotopes. Radiopharmaceuticals can be used as diagnostic and
therapeutic agents. Radiopharmaceuticals emit radiation themselves,
which is different from contrast media which absorb or alter external
electromagnetism or ultrasound. Radiopharmacology is the branch of
pharmacology that specializes in these agents. The main group of these
compounds are the radiotracers used to diagnose dysfunction in body
tissues. While not all medical isotopes are radioactive,
radiopharmaceuticals are the oldest and still most common such drugs.
Radiation therapy was first used to treat cancer more than 100 years
ago. About half of all cancer patients still receive it at some point
during their treatment. And until recently, most radiation therapy was
given much as it was 100 years ago, by delivering beams of radiation
from outside the body to kill tumors inside the body.

\textbf{Week 11}

RPT has proven to be an effective cancer treatment when other standard
therapeutic approaches have failed. However, despite more than 40 years
of clinical investigation, RPT has not become a part of the cancer
treatment armamentarium in the same way as other therapies292.
`Targeted' cancer therapies are associated with clinical trial failure
rates of 97\% (ref.1), partly because the agents targeted a pathway that
was not involved in promoting the cancer phenotype2. By contrast, RPT
has been unsuccessful owing to a failure to adopt and rigorously
evaluate this treatment modality, which may be explained in part by the
multidisciplinary nature of the treatment.

Additional challenges facing the development and application of RPT
include public perception and fear of radioactivity as well as the
perceived complexity of the treatment. Until very recently, the
\textgreater40 years of experience with these agents was largely ignored
or presented as a burdensome multidisciplinary endeavour in the medical
literature. A 2007 review of the management of painful bone
metastases293 highlights this, implying that the efficacy, low toxicity,
minimal side effects and non- addictiveness of RPT for bone pain
palliation is trumped by the complexity and need for a multidisciplinary
implementation. The lack of a medical constituency for RPT suggests the
need for a new specialty or subspecialty to provide the
multidisciplinary training needed to safely and effectively administer
RPT agents to patients and subsequently manage them. Such a specialty or
subspecialty would require training in nuclear medicine, radiation
oncology and also general oncology. As delivery of radiation is
involved, the participation of medical physicists familiar with both
imaging and radionuclide dosimetry is important. Finally, as a radiation
delivery modality, one may envision widespread adoption of treatment
planning that combines RPT with external- beam radiotherapy; the former
to target disseminated cancer and the latter to target bulky disease
that is less effectively treated by RPT. Such a combination strategy
would\textbackslash{} expand patient eligibility for both RPT and
external- beam radiotherapy.

\end{document}
