% Options for packages loaded elsewhere
\PassOptionsToPackage{unicode}{hyperref}
\PassOptionsToPackage{hyphens}{url}
%
\documentclass[
]{article}
\usepackage{lmodern}
\usepackage{amssymb,amsmath}
\usepackage{ifxetex,ifluatex}
\ifnum 0\ifxetex 1\fi\ifluatex 1\fi=0 % if pdftex
  \usepackage[T1]{fontenc}
  \usepackage[utf8]{inputenc}
  \usepackage{textcomp} % provide euro and other symbols
\else % if luatex or xetex
  \usepackage{unicode-math}
  \defaultfontfeatures{Scale=MatchLowercase}
  \defaultfontfeatures[\rmfamily]{Ligatures=TeX,Scale=1}
\fi
% Use upquote if available, for straight quotes in verbatim environments
\IfFileExists{upquote.sty}{\usepackage{upquote}}{}
\IfFileExists{microtype.sty}{% use microtype if available
  \usepackage[]{microtype}
  \UseMicrotypeSet[protrusion]{basicmath} % disable protrusion for tt fonts
}{}
\makeatletter
\@ifundefined{KOMAClassName}{% if non-KOMA class
  \IfFileExists{parskip.sty}{%
    \usepackage{parskip}
  }{% else
    \setlength{\parindent}{0pt}
    \setlength{\parskip}{6pt plus 2pt minus 1pt}}
}{% if KOMA class
  \KOMAoptions{parskip=half}}
\makeatother
\usepackage{xcolor}
\IfFileExists{xurl.sty}{\usepackage{xurl}}{} % add URL line breaks if available
\IfFileExists{bookmark.sty}{\usepackage{bookmark}}{\usepackage{hyperref}}
\hypersetup{
  hidelinks,
  pdfcreator={LaTeX via pandoc}}
\urlstyle{same} % disable monospaced font for URLs
\setlength{\emergencystretch}{3em} % prevent overfull lines
\providecommand{\tightlist}{%
  \setlength{\itemsep}{0pt}\setlength{\parskip}{0pt}}
\setcounter{secnumdepth}{-\maxdimen} % remove section numbering

\author{}
\date{}

\begin{document}

\textbf{{Radiopharmaceuticals: Cancer Therapy}}

Pushkin Rathore 18111043

Week 5

February 18\textsuperscript{th}, 2022

\textbf{AIM}

The aim of this is paper is to make aware the reader about the
technologies that exists that make the treatment of certain diseases
such as cancer to be more efficient by the use of different approaches.
In this particular case, we'll look into the dealings in regards to the
radioactive substances and the science behind them that is being applied
to treat diseases such as cancer. Radiopharmaceuticals, or medicinal
radio compounds, are a group of pharmaceutical drugs containing
radioactive isotopes. Radiopharmaceuticals can be used as diagnostic and
therapeutic agents. Radiopharmaceuticals emit radiation themselves,
which is different from contrast media which absorb or alter external
electromagnetism or ultrasound. Radiopharmacology is the branch of
pharmacology that specializes in these agents. The main group of these
compounds are the radiotracers used to diagnose dysfunction in body
tissues. While not all medical isotopes are radioactive,
radiopharmaceuticals are the oldest and still most common such drugs.
Radiation therapy was first used to treat cancer more than 100 years
ago. About half of all cancer patients still receive it at some point
during their treatment. And until recently, most radiation therapy was
given much as it was 100 years ago, by delivering beams of radiation
from outside the body to kill tumors inside the body.

\textbf{Week 5}

One of the hallmarks of RPT is its ability to deliver highly potent
forms of radiation directly to tumour cells. Three different types of
radiation are relevant to understanding RPT: photons, electrons and α-
particles. Photons come in two `flavours' --- X- rays and γ- rays. The
former are derived from orbital electron transitions and are typically
lower in energy than γ- rays. Radionuclide photon emissions are useful
for imaging the distribution of the RPT but not for localized delivery
of cytotoxic radiation. Although a wide range of photon energies may be
imaged (70--400 keV), photon emission energies in the range from 100 to
200 keV are optimal for all nuclear medicine imaging cameras (γ- cameras
and single- photon emission computed tomography (SPECT) cameras). A
number of radionuclides also emit positrons which lead to the emission
of 511- keV photons that are detected by positron emission tomography
(PET) cameras.

Electron emissions are classified by energy and also by the type of
decay. Auger electrons, β- particles and monoenergetic electrons are
relevant to RPT. Auger electrons are generated from suborbital
transitions. They are typically very- short- range emissions, of the
order of 1--1000 nm, depending on their emission energy. If the RPT drug
localizes within the cell nucleus, these emissions could be highly
cytotoxic. Auger electron- emitter RPT has not been widely adopted,
however. Although preclinical studies have shown substantial therapeutic
efficacy, the small number of human investigations did not lead to
clinical efficacy. Human studies using locoregional administration
showed promise in terms of tumour cell incorporation of the Auger
emitters. The requirement that these agents be incorporated into the DNA
and also their unfavourable pharmacokinetics are thought to be the
reasons underlying the lack of efficacy. Encouraged by ongoing
technological developments that could overcome the factors, these agents
continue to be of interest to the RPT community.

\end{document}
