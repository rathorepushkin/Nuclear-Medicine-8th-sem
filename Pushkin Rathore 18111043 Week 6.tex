% Options for packages loaded elsewhere
\PassOptionsToPackage{unicode}{hyperref}
\PassOptionsToPackage{hyphens}{url}
%
\documentclass[
]{article}
\usepackage{lmodern}
\usepackage{amssymb,amsmath}
\usepackage{ifxetex,ifluatex}
\ifnum 0\ifxetex 1\fi\ifluatex 1\fi=0 % if pdftex
  \usepackage[T1]{fontenc}
  \usepackage[utf8]{inputenc}
  \usepackage{textcomp} % provide euro and other symbols
\else % if luatex or xetex
  \usepackage{unicode-math}
  \defaultfontfeatures{Scale=MatchLowercase}
  \defaultfontfeatures[\rmfamily]{Ligatures=TeX,Scale=1}
\fi
% Use upquote if available, for straight quotes in verbatim environments
\IfFileExists{upquote.sty}{\usepackage{upquote}}{}
\IfFileExists{microtype.sty}{% use microtype if available
  \usepackage[]{microtype}
  \UseMicrotypeSet[protrusion]{basicmath} % disable protrusion for tt fonts
}{}
\makeatletter
\@ifundefined{KOMAClassName}{% if non-KOMA class
  \IfFileExists{parskip.sty}{%
    \usepackage{parskip}
  }{% else
    \setlength{\parindent}{0pt}
    \setlength{\parskip}{6pt plus 2pt minus 1pt}}
}{% if KOMA class
  \KOMAoptions{parskip=half}}
\makeatother
\usepackage{xcolor}
\IfFileExists{xurl.sty}{\usepackage{xurl}}{} % add URL line breaks if available
\IfFileExists{bookmark.sty}{\usepackage{bookmark}}{\usepackage{hyperref}}
\hypersetup{
  hidelinks,
  pdfcreator={LaTeX via pandoc}}
\urlstyle{same} % disable monospaced font for URLs
\setlength{\emergencystretch}{3em} % prevent overfull lines
\providecommand{\tightlist}{%
  \setlength{\itemsep}{0pt}\setlength{\parskip}{0pt}}
\setcounter{secnumdepth}{-\maxdimen} % remove section numbering

\author{}
\date{}

\begin{document}

\textbf{{Radiopharmaceuticals: Cancer Therapy}}

Pushkin Rathore 18111043

Week 6

February 25\textsuperscript{th}, 2022

\textbf{AIM}

The aim of this is paper is to make aware the reader about the
technologies that exists that make the treatment of certain diseases
such as cancer to be more efficient by the use of different approaches.
In this particular case, we'll look into the dealings in regards to the
radioactive substances and the science behind them that is being applied
to treat diseases such as cancer. Radiopharmaceuticals, or medicinal
radio compounds, are a group of pharmaceutical drugs containing
radioactive isotopes. Radiopharmaceuticals can be used as diagnostic and
therapeutic agents. Radiopharmaceuticals emit radiation themselves,
which is different from contrast media which absorb or alter external
electromagnetism or ultrasound. Radiopharmacology is the branch of
pharmacology that specializes in these agents. The main group of these
compounds are the radiotracers used to diagnose dysfunction in body
tissues. While not all medical isotopes are radioactive,
radiopharmaceuticals are the oldest and still most common such drugs.
Radiation therapy was first used to treat cancer more than 100 years
ago. About half of all cancer patients still receive it at some point
during their treatment. And until recently, most radiation therapy was
given much as it was 100 years ago, by delivering beams of radiation
from outside the body to kill tumors inside the body.

\textbf{Week 6}

A variety of reasons for the progression and shift to different
radionuclides may be invoked to explain the changes and additions of the
different β- particle emitters used over time. For example, in an early
theoretical evaluation of different radionuclides, yttrium-90 ranked
second to rhenium-186 in a list of nine radionuclides considered. In
that ranking, the main criterion was the tumour to non- tumour- absorbed
dose ratio. This was a theoretical calculation obtained for different
radionuclides using typical radiolabelled antibody pharmacokinetics for
tumour targeting and organ clearance. This optimum reflected the 64.2-
hour half- life of yttrium-90 and its high- energy β- particle, which
was deemed favourable for uniformly irradiating tumours. Like
iodine-131, the adoption of yttrium-90 for RPT is likely based on its
history and widespread availability. In the 1970s it was used in
colloidal form, primarily to treat rheumatoid conditions. Efforts to
conjugate yttrium-90, a radiometal, were unsuccessful until a radiometal
conjugation chemistry that retained stability in vivo was developed.
Clinical trials using yttrium-90- labelled antibodies as RPT agents
initially focused on ovarian cancer and subsequently on haematological
cancers, as well as radiopeptide therapy. Yttrium-90 continues to be a
popular radionuclide for RPT because of the clinical impact of
yttrium-90- impregnated microspheres that are used for treatment of
hepatic metastases. Although yttrium-90 has been imaged, imaging
generally requires high activities (more than 300 MBq). Such activities
are typically achieved only in microsphere therapies. Lutetium-177
gained popularity because it emits photons in the 100--200- keV optimal
imaging range and has a β- particle energy that is between that of
iodine-131 and yttrium-90, which is appropriate for therapy. These
factors, along with a half- life that is compatible with the
pharmacokinetics of both antibodies and peptides, make this radionuclide
a theranostic in that the same radionuclide may be used to assess tumour
uptake and the extent of cancer, and also as a treatment. It is produced
in a reactor and is therefore widely available, with a relatively
straightforward conjugation chemistry.

\end{document}
