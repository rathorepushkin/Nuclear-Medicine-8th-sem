% Options for packages loaded elsewhere
\PassOptionsToPackage{unicode}{hyperref}
\PassOptionsToPackage{hyphens}{url}
%
\documentclass[
]{article}
\usepackage{lmodern}
\usepackage{amssymb,amsmath}
\usepackage{ifxetex,ifluatex}
\ifnum 0\ifxetex 1\fi\ifluatex 1\fi=0 % if pdftex
  \usepackage[T1]{fontenc}
  \usepackage[utf8]{inputenc}
  \usepackage{textcomp} % provide euro and other symbols
\else % if luatex or xetex
  \usepackage{unicode-math}
  \defaultfontfeatures{Scale=MatchLowercase}
  \defaultfontfeatures[\rmfamily]{Ligatures=TeX,Scale=1}
\fi
% Use upquote if available, for straight quotes in verbatim environments
\IfFileExists{upquote.sty}{\usepackage{upquote}}{}
\IfFileExists{microtype.sty}{% use microtype if available
  \usepackage[]{microtype}
  \UseMicrotypeSet[protrusion]{basicmath} % disable protrusion for tt fonts
}{}
\makeatletter
\@ifundefined{KOMAClassName}{% if non-KOMA class
  \IfFileExists{parskip.sty}{%
    \usepackage{parskip}
  }{% else
    \setlength{\parindent}{0pt}
    \setlength{\parskip}{6pt plus 2pt minus 1pt}}
}{% if KOMA class
  \KOMAoptions{parskip=half}}
\makeatother
\usepackage{xcolor}
\IfFileExists{xurl.sty}{\usepackage{xurl}}{} % add URL line breaks if available
\IfFileExists{bookmark.sty}{\usepackage{bookmark}}{\usepackage{hyperref}}
\hypersetup{
  hidelinks,
  pdfcreator={LaTeX via pandoc}}
\urlstyle{same} % disable monospaced font for URLs
\setlength{\emergencystretch}{3em} % prevent overfull lines
\providecommand{\tightlist}{%
  \setlength{\itemsep}{0pt}\setlength{\parskip}{0pt}}
\setcounter{secnumdepth}{-\maxdimen} % remove section numbering

\author{}
\date{}

\begin{document}

\textbf{{Radiopharmaceuticals: Cancer Therapy}}

Pushkin Rathore 18111043

Week 3

February 4\textsuperscript{th}, 2022

\textbf{AIM}

The aim of this is paper is to make aware the reader about the
technologies that exists that make the treatment of certain diseases
such as cancer to be more efficient by the use of different approaches.
In this particular case, we'll look into the dealings in regards to the
radioactive substances and the science behind them that is being applied
to treat diseases such as cancer. Radiopharmaceuticals, or medicinal
radio compounds, are a group of pharmaceutical drugs containing
radioactive isotopes. Radiopharmaceuticals can be used as diagnostic and
therapeutic agents. Radiopharmaceuticals emit radiation themselves,
which is different from contrast media which absorb or alter external
electromagnetism or ultrasound. Radiopharmacology is the branch of
pharmacology that specializes in these agents. The main group of these
compounds are the radiotracers used to diagnose dysfunction in body
tissues. While not all medical isotopes are radioactive,
radiopharmaceuticals are the oldest and still most common such drugs.
Radiation therapy was first used to treat cancer more than 100 years
ago. About half of all cancer patients still receive it at some point
during their treatment. And until recently, most radiation therapy was
given much as it was 100 years ago, by delivering beams of radiation
from outside the body to kill tumors inside the body.

\textbf{Week 3}

Mechanism and biological effects the mechanism of action for RPT is
radiation- induced killing of cells. Investigation into the effects of
radiation on tissues and tumours began soon after the discovery of
radiation and radioactivity. RPT has the benefit of drawing on the
substantial knowledge base of radiotherapy. However, RPT differs from
radiotherapy, and it is important to understand how those elements
unique to RPT influence therapy. The essential questions for RPT are
where does the agent localize and for how long? As noted in the section
entitled `Dosimetry', answers to these questions inform the tumour
versus normal tissue absorbed dose and provide a measure of potential
treatment success. The biological effects of a given absorbed dose for a
tumour depend on the rate at which the dose is delivered. A dose of 30
Gy delivered to a tumour over a period of many weeks at a dose rate that
is exponentially decreasing, as is typically the case with RPT, will
have a very different effect from that of the same

amount delivered at the much higher dose rates used in radiotherapy (for
example, daily, 2- Gy fractions over 15 days). The difference in
biological outcome will depend on the biological repair and
radiosensitivity properties of the tumour. Dose- rate considerations
also apply to normal organs.

Another fundamental distinguishing feature important for understanding
this treatment modality is the diminishing curative potential with
reduced target cell number. In radiotherapy the probability of killing
all cells for a given absorbed dose increases as the number of target
cells decreases --- fewer cells to kill for a given radiation absorbed
dose increases the chance that all of the cells will be killed. By
contrast, fewer cells do

not translate into a greater tumour control probability in RPT. This is
because the radiation is not delivered uniformly to all cells. If the
emitted radiation originates from a radionuclide on the surface of
tumour cells, fewer cells lead to a smaller fraction of the emitted
energy being deposited into the targeted cells14. This is balanced, in
part, by the greater concentration that may be achieved in smaller
clusters of cells relative to large measurable tumours.

\end{document}
