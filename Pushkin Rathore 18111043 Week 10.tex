% Options for packages loaded elsewhere
\PassOptionsToPackage{unicode}{hyperref}
\PassOptionsToPackage{hyphens}{url}
%
\documentclass[
]{article}
\usepackage{lmodern}
\usepackage{amssymb,amsmath}
\usepackage{ifxetex,ifluatex}
\ifnum 0\ifxetex 1\fi\ifluatex 1\fi=0 % if pdftex
  \usepackage[T1]{fontenc}
  \usepackage[utf8]{inputenc}
  \usepackage{textcomp} % provide euro and other symbols
\else % if luatex or xetex
  \usepackage{unicode-math}
  \defaultfontfeatures{Scale=MatchLowercase}
  \defaultfontfeatures[\rmfamily]{Ligatures=TeX,Scale=1}
\fi
% Use upquote if available, for straight quotes in verbatim environments
\IfFileExists{upquote.sty}{\usepackage{upquote}}{}
\IfFileExists{microtype.sty}{% use microtype if available
  \usepackage[]{microtype}
  \UseMicrotypeSet[protrusion]{basicmath} % disable protrusion for tt fonts
}{}
\makeatletter
\@ifundefined{KOMAClassName}{% if non-KOMA class
  \IfFileExists{parskip.sty}{%
    \usepackage{parskip}
  }{% else
    \setlength{\parindent}{0pt}
    \setlength{\parskip}{6pt plus 2pt minus 1pt}}
}{% if KOMA class
  \KOMAoptions{parskip=half}}
\makeatother
\usepackage{xcolor}
\IfFileExists{xurl.sty}{\usepackage{xurl}}{} % add URL line breaks if available
\IfFileExists{bookmark.sty}{\usepackage{bookmark}}{\usepackage{hyperref}}
\hypersetup{
  hidelinks,
  pdfcreator={LaTeX via pandoc}}
\urlstyle{same} % disable monospaced font for URLs
\setlength{\emergencystretch}{3em} % prevent overfull lines
\providecommand{\tightlist}{%
  \setlength{\itemsep}{0pt}\setlength{\parskip}{0pt}}
\setcounter{secnumdepth}{-\maxdimen} % remove section numbering

\author{}
\date{}

\begin{document}

\textbf{{Radiopharmaceuticals: Cancer Therapy}}

Pushkin Rathore 18111043

Week 10

March 25\textsuperscript{th}, 2022

\textbf{AIM}

The aim of this is paper is to make aware the reader about the
technologies that exists that make the treatment of certain diseases
such as cancer to be more efficient by the use of different approaches.
In this particular case, we'll look into the dealings in regards to the
radioactive substances and the science behind them that is being applied
to treat diseases such as cancer. Radiopharmaceuticals, or medicinal
radio compounds, are a group of pharmaceutical drugs containing
radioactive isotopes. Radiopharmaceuticals can be used as diagnostic and
therapeutic agents. Radiopharmaceuticals emit radiation themselves,
which is different from contrast media which absorb or alter external
electromagnetism or ultrasound. Radiopharmacology is the branch of
pharmacology that specializes in these agents. The main group of these
compounds are the radiotracers used to diagnose dysfunction in body
tissues. While not all medical isotopes are radioactive,
radiopharmaceuticals are the oldest and still most common such drugs.
Radiation therapy was first used to treat cancer more than 100 years
ago. About half of all cancer patients still receive it at some point
during their treatment. And until recently, most radiation therapy was
given much as it was 100 years ago, by delivering beams of radiation
from outside the body to kill tumors inside the body.

\textbf{Week 10}

Nano construct and microsphere RPT Yttrium-90 radioembolization is a
technique that targets radiolabelled microspheres to liver tumours
associated with unresectable hepatocellular carcinomas or metastatic
liver tumours from primary colorectal cancer. Liver tumours derive their
blood supply from the hepatic artery, whereas the normal liver derives
its blood supply from the portal vein, allowing targeted delivery of
90Y- loaded microspheres via intra- arterial injection. The commercially
available 90Y- loaded microspheres are either glass based (TheraSphere)
or resin based (SIR- Spheres), differing in size, number of microspheres
injected and activity per microsphere. Both agents were approved by the
FDA and are marketed as devices. We include these as RPT agents because
they better fit the broad definition of RPT agents provided earlier. The
findings of studies comparing 90Y- loaded glass- based and resin- based
microspheres are conflicting and further investigation is needed.

For example, while one study in patients with unresectable
hepatocellular carcinomas found a significantly higher overall survival
for treatment with 90Y- loaded glass microspheres compared with 90Y-
loaded resin microspheres, another study found a similar outcome in
terms of progression- free survival and overall survival between
patients treated with 90Y- loaded glass- based microspheres and patients
treated with 90Y- loaded resin- based microspheres. Additional
transarterial radiotherapeutics are being explored, including
phosphorus-32 glass microspheres and holmium-166 microspheres as well as
131I- labelled or 188Re- labelled iodized oil. Initial clinical trials
of 131I- labelled iodized oil (131I- labelled Lipiodol) were completed
in the late 1980s/ early 1990s and clinical investigations of this
treatment modality continued until 2013. Administration of 131I-
labelled Lipiodol in the adjuvant setting, after resection or
radiofrequency ablation for hepatocellular carcinoma, yielded a 6- month
increase in recurrencefree survival and a 24- month increase in median
overall survival. In a prospective randomized 43- patient trial,
adjuvant treatment of patients with hepatocellular carcinoma led to a
significant increase in overall survival at 3 years of 86.4\% in the
treated group versus 46.3\% in the control group. At both the 5- year
follow- up and the 10- year follow- up, actuarial overall survival in
the treated group was statistically significantly greater than in the
control group (66.7\% versus 36.4\%, respectively, and 52.4\% versus
27.3\%, respectively). The difference in overall survival lost
statistical significance 8 years after randomization.

\end{document}
