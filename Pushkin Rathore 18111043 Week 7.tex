% Options for packages loaded elsewhere
\PassOptionsToPackage{unicode}{hyperref}
\PassOptionsToPackage{hyphens}{url}
%
\documentclass[
]{article}
\usepackage{lmodern}
\usepackage{amssymb,amsmath}
\usepackage{ifxetex,ifluatex}
\ifnum 0\ifxetex 1\fi\ifluatex 1\fi=0 % if pdftex
  \usepackage[T1]{fontenc}
  \usepackage[utf8]{inputenc}
  \usepackage{textcomp} % provide euro and other symbols
\else % if luatex or xetex
  \usepackage{unicode-math}
  \defaultfontfeatures{Scale=MatchLowercase}
  \defaultfontfeatures[\rmfamily]{Ligatures=TeX,Scale=1}
\fi
% Use upquote if available, for straight quotes in verbatim environments
\IfFileExists{upquote.sty}{\usepackage{upquote}}{}
\IfFileExists{microtype.sty}{% use microtype if available
  \usepackage[]{microtype}
  \UseMicrotypeSet[protrusion]{basicmath} % disable protrusion for tt fonts
}{}
\makeatletter
\@ifundefined{KOMAClassName}{% if non-KOMA class
  \IfFileExists{parskip.sty}{%
    \usepackage{parskip}
  }{% else
    \setlength{\parindent}{0pt}
    \setlength{\parskip}{6pt plus 2pt minus 1pt}}
}{% if KOMA class
  \KOMAoptions{parskip=half}}
\makeatother
\usepackage{xcolor}
\IfFileExists{xurl.sty}{\usepackage{xurl}}{} % add URL line breaks if available
\IfFileExists{bookmark.sty}{\usepackage{bookmark}}{\usepackage{hyperref}}
\hypersetup{
  hidelinks,
  pdfcreator={LaTeX via pandoc}}
\urlstyle{same} % disable monospaced font for URLs
\setlength{\emergencystretch}{3em} % prevent overfull lines
\providecommand{\tightlist}{%
  \setlength{\itemsep}{0pt}\setlength{\parskip}{0pt}}
\setcounter{secnumdepth}{-\maxdimen} % remove section numbering

\author{}
\date{}

\begin{document}

\textbf{{Radiopharmaceuticals: Cancer Therapy}}

Pushkin Rathore 18111043

Week 7

March 4\textsuperscript{th}, 2022

\textbf{AIM}

The aim of this is paper is to make aware the reader about the
technologies that exists that make the treatment of certain diseases
such as cancer to be more efficient by the use of different approaches.
In this particular case, we'll look into the dealings in regards to the
radioactive substances and the science behind them that is being applied
to treat diseases such as cancer. Radiopharmaceuticals, or medicinal
radio compounds, are a group of pharmaceutical drugs containing
radioactive isotopes. Radiopharmaceuticals can be used as diagnostic and
therapeutic agents. Radiopharmaceuticals emit radiation themselves,
which is different from contrast media which absorb or alter external
electromagnetism or ultrasound. Radiopharmacology is the branch of
pharmacology that specializes in these agents. The main group of these
compounds are the radiotracers used to diagnose dysfunction in body
tissues. While not all medical isotopes are radioactive,
radiopharmaceuticals are the oldest and still most common such drugs.
Radiation therapy was first used to treat cancer more than 100 years
ago. About half of all cancer patients still receive it at some point
during their treatment. And until recently, most radiation therapy was
given much as it was 100 years ago, by delivering beams of radiation
from outside the body to kill tumors inside the body.

\textbf{Week 7}

Response and toxicity prediction is essential for the rational
implementation of cancer therapy. The biological effects of radionuclide
therapy are mediated by a well- defined physical quantity, the absorbed
dose (D), which is defined as the energy absorbed per unit mass of
tissue. The long and well- established cancer treatment experience in
radiotherapy has provided ample evidence that absorbed dose may be used
to predict biological response. In chemotherapy, targeted biologic
therapy and immunotherapy, there is no dosimetry analogue. Dosimetry as
implemented in RPT may be thought of as the ability to perform the
equivalent of a pharmacodynamic study in treated patients in real time.
Dosimetry analysis may be performed as part of patient treatment to
calculate tumour versus normal organ absorbed dose and therefore the
likelihood of treatment success. The ability to rapidly assay genetic
and epigenetic characteristics of tumour samples comes closest to
providing the kind of information that RPT dosimetry provides regarding
the potential efficacy and toxicity of a therapeutic agent in an
individual patient.

Current imaging techniques do not possess the resolution required to
resolve activity distributions at the microscopic scale. However, by
pairing whole- organ macroscale measurements that can be performed in
humans with microscale information that can be obtained from preclinical
studies, it is possible to extract microscale information from
macroscale measurements. A contour may be drawn on an image obtained
with a

patient imaging modality such as PET/CT or SPECT/CT that encompasses the
entire organ (for example, kidney) or macroscopic subcompartments within
the organ (for example, renal cortex). These macroscale contours may be
used to obtain time- versus- activity curves (TACs) for the entire organ
or macroscale subcompartments within the organ.

\end{document}
