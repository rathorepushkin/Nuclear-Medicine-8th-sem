% Options for packages loaded elsewhere
\PassOptionsToPackage{unicode}{hyperref}
\PassOptionsToPackage{hyphens}{url}
%
\documentclass[
]{article}
\usepackage{lmodern}
\usepackage{amssymb,amsmath}
\usepackage{ifxetex,ifluatex}
\ifnum 0\ifxetex 1\fi\ifluatex 1\fi=0 % if pdftex
  \usepackage[T1]{fontenc}
  \usepackage[utf8]{inputenc}
  \usepackage{textcomp} % provide euro and other symbols
\else % if luatex or xetex
  \usepackage{unicode-math}
  \defaultfontfeatures{Scale=MatchLowercase}
  \defaultfontfeatures[\rmfamily]{Ligatures=TeX,Scale=1}
\fi
% Use upquote if available, for straight quotes in verbatim environments
\IfFileExists{upquote.sty}{\usepackage{upquote}}{}
\IfFileExists{microtype.sty}{% use microtype if available
  \usepackage[]{microtype}
  \UseMicrotypeSet[protrusion]{basicmath} % disable protrusion for tt fonts
}{}
\makeatletter
\@ifundefined{KOMAClassName}{% if non-KOMA class
  \IfFileExists{parskip.sty}{%
    \usepackage{parskip}
  }{% else
    \setlength{\parindent}{0pt}
    \setlength{\parskip}{6pt plus 2pt minus 1pt}}
}{% if KOMA class
  \KOMAoptions{parskip=half}}
\makeatother
\usepackage{xcolor}
\IfFileExists{xurl.sty}{\usepackage{xurl}}{} % add URL line breaks if available
\IfFileExists{bookmark.sty}{\usepackage{bookmark}}{\usepackage{hyperref}}
\hypersetup{
  hidelinks,
  pdfcreator={LaTeX via pandoc}}
\urlstyle{same} % disable monospaced font for URLs
\setlength{\emergencystretch}{3em} % prevent overfull lines
\providecommand{\tightlist}{%
  \setlength{\itemsep}{0pt}\setlength{\parskip}{0pt}}
\setcounter{secnumdepth}{-\maxdimen} % remove section numbering

\author{}
\date{}

\begin{document}

\textbf{{Radiopharmaceuticals: Cancer Therapy}}

Pushkin Rathore 18111043

Week 1

January 20\textsuperscript{th}, 2022

\textbf{AIM}

The aim of this is paper is to make aware the reader about the
technologies that exists that make the treatment of certain diseases
such as cancer to be more efficient by the use of different approaches.
In this particular case, we'll look into the dealings in regards to the
radioactive substances and the science behind them that is being applied
to treat diseases such as cancer. Radiopharmaceuticals, or medicinal
radio compounds, are a group of pharmaceutical drugs containing
radioactive isotopes. Radiopharmaceuticals can be used as diagnostic and
therapeutic agents. Radiopharmaceuticals emit radiation themselves,
which is different from contrast media which absorb or alter external
electromagnetism or ultrasound. Radiopharmacology is the branch of
pharmacology that specializes in these agents. The main group of these
compounds are the radiotracers used to diagnose dysfunction in body
tissues. While not all medical isotopes are radioactive,
radiopharmaceuticals are the oldest and still most common such drugs.
Radiation therapy was first used to treat cancer more than 100 years
ago. About half of all cancer patients still receive it at some point
during their treatment. And until recently, most radiation therapy was
given much as it was 100 years ago, by delivering beams of radiation
from outside the body to kill tumors inside the body.

\textbf{Week 1}

Delivering radiation directly to cells isn't itself a new approach. One
such therapy, called radioactive iodine, has been used to treat some
types of thyroid cancer since the 1940s. Iodine naturally accumulates in
thyroid cells. A radioactive version of the element can be produced in
the lab. When ingested (as a pill or a liquid), it accumulates in and
kills cancer cells left over after thyroid surgery.

A similar natural affinity was later exploited to develop drugs to treat
cancer that has spread to the bones, such as radium 223 dichloride,
which was approved in 2013 to treat metastatic prostate cancer. When
cancer cells grow in the bone, they cause the bone tissue they invade to
break down. The body then attempts to repair this damage by replacing
that bone---a process called bone turnover. The radioactive element
radium ``looks like a calcium molecule, so it gets incorporated into
areas of the body where bone turnover is highest,'' such as areas where
cancer is growing. The radium is then able to kill nearby cancer cells.
These radioactive compounds all travel to cancer cells without any help.
Researchers wondered whether it would be possible to engineer new
radioactive molecules that specifically target other cancers.

Radiopharmaceuticals can be divided into four categories:

\textbf{Radiopharmaceutical preparation} A radiopharmaceutical
preparation is a medicinal product in a ready-to-use form suitable for
human use that contains a radionuclide. The radionuclide is integral to
the medicinal application of the preparation, making it appropriate for
one or more diagnostic or therapeutic applications.

\textbf{Radionuclide generator} A system in which a daughter
radionuclide (short half-life) is separated by elution or by other means
from a parent radionuclide (long half-life) and later used for
production of a radiopharmaceutical preparation.

\textbf{Radiopharmaceutical precursor} A radionuclide produced for the
radiolabelling process with a resultant radiopharmaceutical preparation.

\textbf{Kit for radiopharmaceutical preparation} In general a vial
containing the nonradionuclide components of a radiopharmaceutical
preparation , usually in the form of a sterilized, validated product to
which the appropriate radionuclide is added or in which the appropriate
radionuclide is diluted before medical use. In most cases the kit is a
multidose vial and production of the radiopharmaceutical preparation may
require additional steps such as boiling, heating, filtration and
buffering. Radiopharmaceutical preparations derived from kits are
normally intended for use within 12 hours of preparation.

\end{document}
