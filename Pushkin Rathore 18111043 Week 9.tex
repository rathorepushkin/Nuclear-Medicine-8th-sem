% Options for packages loaded elsewhere
\PassOptionsToPackage{unicode}{hyperref}
\PassOptionsToPackage{hyphens}{url}
%
\documentclass[
]{article}
\usepackage{lmodern}
\usepackage{amssymb,amsmath}
\usepackage{ifxetex,ifluatex}
\ifnum 0\ifxetex 1\fi\ifluatex 1\fi=0 % if pdftex
  \usepackage[T1]{fontenc}
  \usepackage[utf8]{inputenc}
  \usepackage{textcomp} % provide euro and other symbols
\else % if luatex or xetex
  \usepackage{unicode-math}
  \defaultfontfeatures{Scale=MatchLowercase}
  \defaultfontfeatures[\rmfamily]{Ligatures=TeX,Scale=1}
\fi
% Use upquote if available, for straight quotes in verbatim environments
\IfFileExists{upquote.sty}{\usepackage{upquote}}{}
\IfFileExists{microtype.sty}{% use microtype if available
  \usepackage[]{microtype}
  \UseMicrotypeSet[protrusion]{basicmath} % disable protrusion for tt fonts
}{}
\makeatletter
\@ifundefined{KOMAClassName}{% if non-KOMA class
  \IfFileExists{parskip.sty}{%
    \usepackage{parskip}
  }{% else
    \setlength{\parindent}{0pt}
    \setlength{\parskip}{6pt plus 2pt minus 1pt}}
}{% if KOMA class
  \KOMAoptions{parskip=half}}
\makeatother
\usepackage{xcolor}
\IfFileExists{xurl.sty}{\usepackage{xurl}}{} % add URL line breaks if available
\IfFileExists{bookmark.sty}{\usepackage{bookmark}}{\usepackage{hyperref}}
\hypersetup{
  hidelinks,
  pdfcreator={LaTeX via pandoc}}
\urlstyle{same} % disable monospaced font for URLs
\setlength{\emergencystretch}{3em} % prevent overfull lines
\providecommand{\tightlist}{%
  \setlength{\itemsep}{0pt}\setlength{\parskip}{0pt}}
\setcounter{secnumdepth}{-\maxdimen} % remove section numbering

\author{}
\date{}

\begin{document}

\textbf{{Radiopharmaceuticals: Cancer Therapy}}

Pushkin Rathore 18111043

Week 9

March 18\textsuperscript{th}, 2022

\textbf{AIM}

The aim of this is paper is to make aware the reader about the
technologies that exists that make the treatment of certain diseases
such as cancer to be more efficient by the use of different approaches.
In this particular case, we'll look into the dealings in regards to the
radioactive substances and the science behind them that is being applied
to treat diseases such as cancer. Radiopharmaceuticals, or medicinal
radio compounds, are a group of pharmaceutical drugs containing
radioactive isotopes. Radiopharmaceuticals can be used as diagnostic and
therapeutic agents. Radiopharmaceuticals emit radiation themselves,
which is different from contrast media which absorb or alter external
electromagnetism or ultrasound. Radiopharmacology is the branch of
pharmacology that specializes in these agents. The main group of these
compounds are the radiotracers used to diagnose dysfunction in body
tissues. While not all medical isotopes are radioactive,
radiopharmaceuticals are the oldest and still most common such drugs.
Radiation therapy was first used to treat cancer more than 100 years
ago. About half of all cancer patients still receive it at some point
during their treatment. And until recently, most radiation therapy was
given much as it was 100 years ago, by delivering beams of radiation
from outside the body to kill tumors inside the body.

\textbf{Week 9}

The success of iodide-131 in targeting and treating thyroid disorders
and carcinomas encouraged the expansion of its use in a variety of
cancers through its incorporation into targeting vectors. For example,
iobenguane I-131 is the radioiodinated small- molecule meta-
iodobenzylguanidine ({[}131I{]}mIBG), an analogue of the adrenergic
neutrotransmitter noradrenaline that is used to treat patients with
neuroblastomas140 142. Iodide-131 can be introduced to targeting vectors
as a highly reactive electrophilic iodine compound, allowing rapid
iodination of molecules containing activated aromatic groups, or through
displacement by nucleophilic attack of the radioiodide143. mIBG
radiolabelled with high- specific- activity iodine-131 was recently
approved by the FDA for the treatment of adult and paediatric patients
aged 12 years or older with unresectable metastatic phaeochromocytoma or
paraganglioma. No FDA approved therapy was available for these
conditions before approval of this agent. Use of this agent requires a
positive mIBG imaging scan, standard, weight- based therapeutic dosing
and the application of a process for individualized dosimetry using a
pretreatment tracer study to calculate absorbed doses for normal organs.
Normal organ dosimetry is used to adjust the activity administered so
that the organ doses are below specified threshold levels. FDA approval
of this agent was based on the substantial pre- existing experience with
{[}131I{]}mIBG144--151 and on a recent phase I study which yielded 1-
year and 2- year overall survival of 85.7\% and 61.9\%, respectively, in
21 patients treated with the maximum tolerated dose152. Clinical trials
using this agent are

ongoing (NCT03561259 and NCT02378428).

\end{document}
