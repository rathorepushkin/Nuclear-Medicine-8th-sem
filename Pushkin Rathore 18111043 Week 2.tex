% Options for packages loaded elsewhere
\PassOptionsToPackage{unicode}{hyperref}
\PassOptionsToPackage{hyphens}{url}
%
\documentclass[
]{article}
\usepackage{lmodern}
\usepackage{amssymb,amsmath}
\usepackage{ifxetex,ifluatex}
\ifnum 0\ifxetex 1\fi\ifluatex 1\fi=0 % if pdftex
  \usepackage[T1]{fontenc}
  \usepackage[utf8]{inputenc}
  \usepackage{textcomp} % provide euro and other symbols
\else % if luatex or xetex
  \usepackage{unicode-math}
  \defaultfontfeatures{Scale=MatchLowercase}
  \defaultfontfeatures[\rmfamily]{Ligatures=TeX,Scale=1}
\fi
% Use upquote if available, for straight quotes in verbatim environments
\IfFileExists{upquote.sty}{\usepackage{upquote}}{}
\IfFileExists{microtype.sty}{% use microtype if available
  \usepackage[]{microtype}
  \UseMicrotypeSet[protrusion]{basicmath} % disable protrusion for tt fonts
}{}
\makeatletter
\@ifundefined{KOMAClassName}{% if non-KOMA class
  \IfFileExists{parskip.sty}{%
    \usepackage{parskip}
  }{% else
    \setlength{\parindent}{0pt}
    \setlength{\parskip}{6pt plus 2pt minus 1pt}}
}{% if KOMA class
  \KOMAoptions{parskip=half}}
\makeatother
\usepackage{xcolor}
\IfFileExists{xurl.sty}{\usepackage{xurl}}{} % add URL line breaks if available
\IfFileExists{bookmark.sty}{\usepackage{bookmark}}{\usepackage{hyperref}}
\hypersetup{
  hidelinks,
  pdfcreator={LaTeX via pandoc}}
\urlstyle{same} % disable monospaced font for URLs
\setlength{\emergencystretch}{3em} % prevent overfull lines
\providecommand{\tightlist}{%
  \setlength{\itemsep}{0pt}\setlength{\parskip}{0pt}}
\setcounter{secnumdepth}{-\maxdimen} % remove section numbering

\author{}
\date{}

\begin{document}

\textbf{{Radiopharmaceuticals: Cancer Therapy}}

Pushkin Rathore 18111043

Week 2

January 27\textsuperscript{th}, 2022

\textbf{AIM}

The aim of this is paper is to make aware the reader about the
technologies that exists that make the treatment of certain diseases
such as cancer to be more efficient by the use of different approaches.
In this particular case, we'll look into the dealings in regards to the
radioactive substances and the science behind them that is being applied
to treat diseases such as cancer. Radiopharmaceuticals, or medicinal
radio compounds, are a group of pharmaceutical drugs containing
radioactive isotopes. Radiopharmaceuticals can be used as diagnostic and
therapeutic agents. Radiopharmaceuticals emit radiation themselves,
which is different from contrast media which absorb or alter external
electromagnetism or ultrasound. Radiopharmacology is the branch of
pharmacology that specializes in these agents. The main group of these
compounds are the radiotracers used to diagnose dysfunction in body
tissues. While not all medical isotopes are radioactive,
radiopharmaceuticals are the oldest and still most common such drugs.
Radiation therapy was first used to treat cancer more than 100 years
ago. About half of all cancer patients still receive it at some point
during their treatment. And until recently, most radiation therapy was
given much as it was 100 years ago, by delivering beams of radiation
from outside the body to kill tumors inside the body.

\textbf{Week 1}

In this week, we'll look into the traditional methodologies implied for
the treatment of the vile disease called cancer, and too we'll derive
the conclusion why these methodologies are not at par with
radiopharmaceuticals. There are many types of cancer treatment. The
types of treatment that you receive will depend on the type of cancer
you have and how advanced it is. Some people with cancer will have only
one treatment. But most people have a combination of treatments, such as
surgery with chemotherapy and radiation therapy. When you need treatment
for cancer, you have a lot to learn and think about. It is normal to
feel overwhelmed and confused. But, talking with your doctor and
learning about the types of treatment you may have can help you feel
more in control.

\textbf{Biomarker testing} is a way to look for genes, proteins, and
other substances (called biomarkers or tumor markers) that can provide
information about cancer. Biomarker testing can help you and your doctor
choose a cancer treatment.

\textbf{Chemotherapy} is a type of cancer treatment that uses drugs to
kill cancer cells. Learn how chemotherapy works against cancer, why it
causes side effects, and how it is used with other cancer treatments.

\textbf{Hormone therapy} is a treatment that slows or stops the growth
of breast and prostate cancers that use hormones to grow. Learn about
the types of hormone therapy and side effects that may happen.

\textbf{Hyperthermia} is a type of treatment in which body tissue is
heated to as high as 113 °F to help damage and kill cancer cells with
little or no harm to normal tissue. Learn about the types of cancer and
precancers that hyperthermia is used to treat, how it is given, and the
benefits and drawbacks of using hyperthermia.

\textbf{Immunotherapy} is a type of cancer treatment that helps your
immune system fight cancer. This page covers the types of immunotherapy,
how it is used against cancer, and what you can expect during treatment.

\textbf{Photodynamic therapy} uses a drug activated by light to kill
cancer and other abnormal cells. Learn how photodynamic therapy works,
about the types of cancer and precancers it is used to treat, and the
benefits and drawbacks of this treatment.

\textbf{Radiation therapy} is a type of cancer treatment that uses high
doses of radiation to kill cancer cells and shrink tumors. Learn about
the types of radiation, why side effects happen, which side effects you
might have, and more.

\textbf{Stem cell transplants} are procedures that restore stem cells
that grow into blood cells in people who have had theirs destroyed by
high doses of chemotherapy or radiation therapy. Learn about the types
of transplants, side effects that may occur, and how stem cell
transplants are used in cancer treatment.

\textbf{Surgery}, When used to treat cancer, surgery is a procedure in
which a surgeon removes cancer from your body. Learn the different ways
that surgery is used against cancer and what you can expect before,
during, and after surgery.

\textbf{Targeted therapy} is a type of cancer treatment that targets the
changes in cancer cells that help them grow, divide, and spread. Learn
how targeted therapy works against cancer and about common side effects
that may occur.

\end{document}
