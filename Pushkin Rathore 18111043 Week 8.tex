% Options for packages loaded elsewhere
\PassOptionsToPackage{unicode}{hyperref}
\PassOptionsToPackage{hyphens}{url}
%
\documentclass[
]{article}
\usepackage{lmodern}
\usepackage{amssymb,amsmath}
\usepackage{ifxetex,ifluatex}
\ifnum 0\ifxetex 1\fi\ifluatex 1\fi=0 % if pdftex
  \usepackage[T1]{fontenc}
  \usepackage[utf8]{inputenc}
  \usepackage{textcomp} % provide euro and other symbols
\else % if luatex or xetex
  \usepackage{unicode-math}
  \defaultfontfeatures{Scale=MatchLowercase}
  \defaultfontfeatures[\rmfamily]{Ligatures=TeX,Scale=1}
\fi
% Use upquote if available, for straight quotes in verbatim environments
\IfFileExists{upquote.sty}{\usepackage{upquote}}{}
\IfFileExists{microtype.sty}{% use microtype if available
  \usepackage[]{microtype}
  \UseMicrotypeSet[protrusion]{basicmath} % disable protrusion for tt fonts
}{}
\makeatletter
\@ifundefined{KOMAClassName}{% if non-KOMA class
  \IfFileExists{parskip.sty}{%
    \usepackage{parskip}
  }{% else
    \setlength{\parindent}{0pt}
    \setlength{\parskip}{6pt plus 2pt minus 1pt}}
}{% if KOMA class
  \KOMAoptions{parskip=half}}
\makeatother
\usepackage{xcolor}
\IfFileExists{xurl.sty}{\usepackage{xurl}}{} % add URL line breaks if available
\IfFileExists{bookmark.sty}{\usepackage{bookmark}}{\usepackage{hyperref}}
\hypersetup{
  hidelinks,
  pdfcreator={LaTeX via pandoc}}
\urlstyle{same} % disable monospaced font for URLs
\setlength{\emergencystretch}{3em} % prevent overfull lines
\providecommand{\tightlist}{%
  \setlength{\itemsep}{0pt}\setlength{\parskip}{0pt}}
\setcounter{secnumdepth}{-\maxdimen} % remove section numbering

\author{}
\date{}

\begin{document}

\textbf{{Radiopharmaceuticals: Cancer Therapy}}

Pushkin Rathore 18111043

Week 8

March 11\textsuperscript{th}, 2022

\textbf{AIM}

The aim of this is paper is to make aware the reader about the
technologies that exists that make the treatment of certain diseases
such as cancer to be more efficient by the use of different approaches.
In this particular case, we'll look into the dealings in regards to the
radioactive substances and the science behind them that is being applied
to treat diseases such as cancer. Radiopharmaceuticals, or medicinal
radio compounds, are a group of pharmaceutical drugs containing
radioactive isotopes. Radiopharmaceuticals can be used as diagnostic and
therapeutic agents. Radiopharmaceuticals emit radiation themselves,
which is different from contrast media which absorb or alter external
electromagnetism or ultrasound. Radiopharmacology is the branch of
pharmacology that specializes in these agents. The main group of these
compounds are the radiotracers used to diagnose dysfunction in body
tissues. While not all medical isotopes are radioactive,
radiopharmaceuticals are the oldest and still most common such drugs.
Radiation therapy was first used to treat cancer more than 100 years
ago. About half of all cancer patients still receive it at some point
during their treatment. And until recently, most radiation therapy was
given much as it was 100 years ago, by delivering beams of radiation
from outside the body to kill tumors inside the body.

\textbf{Week 8}

In principle, RPT may be applied to any cancer that satisfies the
targeting criteria needed for delivery of radionuclides. However, RPT
has been investigated for only selected cancers. The type of cancer
investigated reflects developments related to the available targets, the
availability of RPT agents against the targets, and the expertise and
clinical investigators at academic institutions. RPT has had the
greatest historical impact for thyroid malignancies and this persists to
the present day. Haematological malignancies were investigated starting
in the early 1990s and continue to be a subject of interest. RPT for
hepatic malignancies and prostate cancer has seen the greatest increase
since the 1980s.

This increase is consistent with the development of new RPT agents, 90Y-
loaded microspheres and β- emitter- labelled and α- emitter- labelled
small- molecule prostate- specific membrane antigen (PSMA)- targeting
constructs, respectively (see later). The FDA- approved α- emitter 223Ra
has also driven the substantial increase in interest in RPT for prostate
cancer. Other solid cancers such as colorectal and breast cancer
continue to be of interest but have not had the breakthrough construct
development that has driven interest in RPT in hepatic and prostate
cancer. Neuroendocrine and somatostatin receptor cancers have been an
ongoing subject of investigation, and the RPT agents targeting these
cancers have probably reached maturity with the FDA approval. A number
of RPT agents are currently on the market, with many more in
development. These include four β- particle and five α- particle
emitters. Lead-212 decays to bismuth-212 and is used as a means to
deliver 212Bi, an α- emitter, without being constrained by its 1- hour
half- life. Of the 30 RPT agents deliver radionuclides that decay by α-
particle emission. The interest in α- emitters reflects a potential
growth area in RPT.

\end{document}
