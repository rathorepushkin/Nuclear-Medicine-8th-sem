% Options for packages loaded elsewhere
\PassOptionsToPackage{unicode}{hyperref}
\PassOptionsToPackage{hyphens}{url}
%
\documentclass[
]{article}
\usepackage{lmodern}
\usepackage{amssymb,amsmath}
\usepackage{ifxetex,ifluatex}
\ifnum 0\ifxetex 1\fi\ifluatex 1\fi=0 % if pdftex
  \usepackage[T1]{fontenc}
  \usepackage[utf8]{inputenc}
  \usepackage{textcomp} % provide euro and other symbols
\else % if luatex or xetex
  \usepackage{unicode-math}
  \defaultfontfeatures{Scale=MatchLowercase}
  \defaultfontfeatures[\rmfamily]{Ligatures=TeX,Scale=1}
\fi
% Use upquote if available, for straight quotes in verbatim environments
\IfFileExists{upquote.sty}{\usepackage{upquote}}{}
\IfFileExists{microtype.sty}{% use microtype if available
  \usepackage[]{microtype}
  \UseMicrotypeSet[protrusion]{basicmath} % disable protrusion for tt fonts
}{}
\makeatletter
\@ifundefined{KOMAClassName}{% if non-KOMA class
  \IfFileExists{parskip.sty}{%
    \usepackage{parskip}
  }{% else
    \setlength{\parindent}{0pt}
    \setlength{\parskip}{6pt plus 2pt minus 1pt}}
}{% if KOMA class
  \KOMAoptions{parskip=half}}
\makeatother
\usepackage{xcolor}
\IfFileExists{xurl.sty}{\usepackage{xurl}}{} % add URL line breaks if available
\IfFileExists{bookmark.sty}{\usepackage{bookmark}}{\usepackage{hyperref}}
\hypersetup{
  hidelinks,
  pdfcreator={LaTeX via pandoc}}
\urlstyle{same} % disable monospaced font for URLs
\setlength{\emergencystretch}{3em} % prevent overfull lines
\providecommand{\tightlist}{%
  \setlength{\itemsep}{0pt}\setlength{\parskip}{0pt}}
\setcounter{secnumdepth}{-\maxdimen} % remove section numbering

\author{}
\date{}

\begin{document}

\textbf{{Radiopharmaceuticals: Cancer Therapy}}

Pushkin Rathore 18111043

Week 3

February 4\textsuperscript{th}, 2022

\textbf{AIM}

The aim of this is paper is to make aware the reader about the
technologies that exists that make the treatment of certain diseases
such as cancer to be more efficient by the use of different approaches.
In this particular case, we'll look into the dealings in regards to the
radioactive substances and the science behind them that is being applied
to treat diseases such as cancer. Radiopharmaceuticals, or medicinal
radio compounds, are a group of pharmaceutical drugs containing
radioactive isotopes. Radiopharmaceuticals can be used as diagnostic and
therapeutic agents. Radiopharmaceuticals emit radiation themselves,
which is different from contrast media which absorb or alter external
electromagnetism or ultrasound. Radiopharmacology is the branch of
pharmacology that specializes in these agents. The main group of these
compounds are the radiotracers used to diagnose dysfunction in body
tissues. While not all medical isotopes are radioactive,
radiopharmaceuticals are the oldest and still most common such drugs.
Radiation therapy was first used to treat cancer more than 100 years
ago. About half of all cancer patients still receive it at some point
during their treatment. And until recently, most radiation therapy was
given much as it was 100 years ago, by delivering beams of radiation
from outside the body to kill tumors inside the body.

\textbf{Week 3}

This week we'll look into one of the traditional forms or methodologies
of treatment of cancer, termed as Chemotherapy, and too we shall
understand the drawbacks of this methods compared to nuclear medicine
treatment. Chemotherapy (often abbreviated to chemo and sometimes CTX or
CTx) is a type of cancer treatment that uses one or more anti-cancer
drugs (chemotherapeutic agents) as part of a standardized chemotherapy
regimen. Chemotherapy may be given with a curative intent (which almost
always involves combinations of drugs), or it may aim to prolong life or
to reduce symptoms (palliative chemotherapy). Chemotherapy is one of the
major categories of the medical discipline specifically devoted to
pharmacotherapy for cancer, which is called medical oncology. The term
chemotherapy has come to connote non-specific usage of intracellular
poisons to inhibit mitosis (cell division) or induce DNA damage, which
is why inhibition of DNA repair can augment chemotherapy. The
connotation of the word chemotherapy excludes more selective agents that
block extracellular signals (signal transduction). The development of
therapies with specific molecular or genetic targets, which inhibit
growth-promoting signals from classic endocrine hormones (primarily
estrogens for breast cancer and androgens for prostate cancer) are now
called hormonal therapies. By contrast, other inhibitions of
growth-signals like those associated with receptor tyrosine kinases are
referred to as targeted therapy. Dosage of chemotherapy can be
difficult: If the dose is too low, it will be ineffective against the
tumor, whereas, at excessive doses, the toxicity (side-effects) will be
intolerable to the person receiving it. The standard method of
determining chemotherapy dosage is based on calculated body surface area
(BSA). The BSA is usually calculated with a mathematical formula or a
nomogram, using the recipient's weight and height, rather than by direct
measurement of body area. This formula was originally derived in a 1916
study and attempted to translate medicinal doses established with
laboratory animals to equivalent doses for humans. The study only
included nine human subjects. When chemotherapy was introduced in the
1950s, the BSA formula was adopted as the official standard for
chemotherapy dosing for lack of a better option. The efficiency of
chemotherapy depends on the type of cancer and the stage. The overall
effectiveness ranges from being curative for some cancers, such as some
leukaemia's, to being ineffective, such as in some brain tumors, to
being needless in others, like most non-melanoma skin cancers.

Chemotherapy does not always work, and even when it is useful, it may
not completely destroy the cancer. People frequently fail to understand
its limitations. In one study of people who had been newly diagnosed
with incurable, stage 4 cancer, more than two-thirds of people with lung
cancer and more than four-fifths of people with colorectal cancer still
believed that chemotherapy was likely to cure their cancer.

\end{document}
